\documentclass{article}
\usepackage{amsmath}
\usepackage{amsfonts}
\usepackage{amssymb}

\newenvironment{proof}{\paragraph{Proof:}}{\hfill$\square$}
\newtheorem{theorem}{Theorem}
\newtheorem{lemma}[theorem]{Lemma}
\newtheorem{corollary}[theorem]{Corollary}

\newcommand{\A}{\mathcal{A}}
\newcommand{\F}{\mathcal{F}}
\newcommand{\prob}{\boldsymbol{P}}

\newcommand{\N}{\mathbb{N}}

\author{Arthur Chen}
\title{Chapter 1}
\date{\today}

\begin{document}
\maketitle

\section*{1.1 Probability spaces, measures, and $\sigma$-algebras}

\subsection*{Problem 1.1.4}

Let $(\Omega, \F, \prob)$ be a probability space and $A, B, A_i$ events in $\F$. Prove the following properties of every probability measure.

\subsubsection*{Part a}

Monotonicity: If $A \subset B$ then $\prob(A) \leq \prob(B)$.

If $A$ is not a proper subset of $B$, then $A=B$ and equality follows because $\prob$ is a function. Otherwise, $A$ partitions $B$ into $A$ and $B \backslash A$. Then by countable subadditivity,

\[
\prob(B) = \prob(A) + \prob(B \backslash A) \geq \prob(A)
\]

by the nonnegativity of measure.

\subsubsection*{Part b}

Sub-additivity:

\[
A \subset \cup_i A_i \longrightarrow \prob(A) \leq \sum_i \prob(A_i)
\]

Let $B_i = \cup_{k=1}^i A_k$, and $C_i = B_i - B_{i-1}$, letting $B_0 = \emptyset$. Then $\cup_i A_i = \cup_i C_i$, and $A \subset \cup_i C_i$. By monotonicity, $\prob(A) \leq \prob(\cup_i C_i)$. Since the $C_i$ are disjoint, $\prob(\cup_i C_i) = \sum_{i=1} \prob(C_i)$. Since $C_i = \cup_{j=1}^i A_j - \cup_{k=1}^{i-1}A_k$, $C_i \subset A_i$, so by monotonicity,

\[
\prob(A) \leq \sum_{i=1} \prob(C_i) \leq \sum_{i=1} \prob(A_i)
\]

\subsubsection*{Part c}

Continuity from Below:

$A_i \uparrow A$ implies $\prob(A_i) \uparrow \prob(A)$.

Note that $\prob(A_i)$ is weakly increasing.

Define $B_i = A_i - A_{i-1}$, and let $A_0 = \emptyset$. Then $A_i = \cup_{k=1}^i B_i$. Then by disjointedness $\prob(A_i) = \sum_{k=1}^i \prob(B_k)$ implies

\[
A = \cup_{i=1}^\infty = \cup_{i=1}^\infty \cup_{k=1}^i B_k = \cup_{i=1}^\infty
\]

and by countable additivity,

\[
\prob(A) = \sum_{k=1}^\infty \prob(B_k)
\]

Since the range of $\prob$ is $[0, 1]$, $|\prob(A)| < \infty$ and the series $\sum_{k=1}^\infty \prob(B_k)$ converges to $\prob(A)$. Since $A_i = \cup_{k=1}^i B_k$, $\prob(A_i)$ is the $i$th partial sum of the $\prob(B_k)$ series, which converges. Thus the $\prob(A_i)$ sequence converges to $\prob(A)$.

\subsubsection*{Part d}

Continuity from Above:

$A_i \downarrow A, \cap_i^\infty A_i = A$ implies $\prob(A_i) \downarrow A$.

Consider $A_i^C$. The $A_i^C$ are increasing sets because the $A_i$ are decreasing. $\cap_{i=1}^\infty A_i = A$ implies $A^C = (\cap_{i=1}^\infty A_i)^C = \cup_{i=1}^\infty A_i^C$. By continuity from below, $\prob(A_i^C) \uparrow \prob(A^C)$. Because $\prob(\Omega) = 1$, this implies $\prob(A_i) \downarrow \prob(A)$ as desired. 

\subsection*{Problem 1.1.5}

Prove that a finitely additive non-negative set function $\mu$ on a measurable space $(\Omega, \F)$ with the 'continuity' property

\[
B_n \subset \F, B_n \downarrow \emptyset, \mu(B_n) < \infty \longrightarrow \mu(B_n) \rightarrow 0
\]

must be countably additive if $\mu(\Omega) < \infty$. Give an example that it is not necessarily so when $\mu(\Omega) = \infty$.

Let $A_n \in \F$ be a countable collection of disjoint, measurable sets. Let $B_k = \cup_{i=k}^\infty A_i$.

Since sigma-algebras are closed under countable union, $B_k \in \F$ for all $k \in \N$. Since $\mu(\Omega) < \infty$, monotonicity implies that $\mu(B_k) < \infty$.

\begin{lemma}
$B_k \downarrow \emptyset$.

\begin{proof}
The limit suupremum of $B_k$ is the set of points $x$ such that $x \in B_k$ for infinitely many $k$. I will show this set is empty.

$B_k$ is the union of disjoint sets $A_i$, so $x \in B_k$ for some $k$ implies that $x \in A_j$ for some $j \in \N$. This then implies that $x \in B_k$ for $k \leq k$, and $x \not\in B_k$ for $k > j$. Thus $x$ is only in a finite number of $B_k$. Thus $\limsup B_k = \emptyset$. Since $\liminf B_k \subset \limsup B_k$ and the only subset of the empty set is the empty set, $\limsup B_k = \emptyset$. Since the upper and lower limit equal each other, $\lim B_k = \emptyset$. Since the $B_k$ are weakly decreasing, the lemma is proved.
\end{proof}
\end{lemma}

We will now prove the main result. For all $n \in \N$, finite additivity means

\[
\prob(\cup_{i=1}^\infty A_i) = \sum_{i=1}^n \prob(A_i) + \prob(\cup_{i=n+1}^\infty A_i) = \sum_{i=1}^n \prob(A_i) + \prob(B_{n+1})
\]

Taking limits, the last term goes to zero, leaving

\[
\prob(\cup_{i=1}^\infty A_i) = \sum_{i=1}^\infty \prob(A_i)
\]

as desired.

For an example of how this breaks down when $\mu(\Omega) = \infty$, let $\Omega = \N$, $\F = 2^\N$, and $\mu$ be finitely-additive with $\mu(\emptyset) = 0$ and the following properties:

\begin{itemize}
\item If $A_n \in \F$ is an event such that $A$ is a singleton that only contains $n \in N$, then $\mu(A_n) = \frac{1}{2^n}$.
\item If $A \in \F$ is infinite, $\mu(A) = \infty$.
\end{itemize}

Thus, $\mu(\cup_{i=1}^\infty A_i) = \mu(\Omega) = \infty$, but $\lim_{n \rightarrow \infty} \sum_{i=1}^n \mu(A_i) = 2$. Thus $\mu$ is not countably additive.

\subsection*{Problem 1.1.8}

Suppose $\prob$ is non-atomic and $A \in F$ with $\prob(A) > 0$.

\subsubsection*{Part a}

Show that for every $\epsilon > 0$, we have $B \subset A$ such that $0 < \prob(B) < \epsilon$.

If $\epsilon \geq \prob(A)$, then the proof is trivial. Otherwise, we will construct a sequence of decreasing subsets $A_n$ such that eventually, $\prob(A_n) < \epsilon$. Let $A_1 = A$. Because $\prob$ is nonatomic, there exists $B_1 \subset A, B_1 \in \F$ such that $\prob(B_1) < \prob(A_1)$. Since sigma-algebras are closed under complements, $A_1 \backslash B_1$ is also measurable, with $\prob(B_1) + \prob(A_1 \backslash B_1) = \prob(A_1)$. Because of this, we know that $\min(\prob(B_1), \prob(A_1 \backslash B_1)) \leq \frac{1}{2} \prob(A_1)$. Denote the subset with the smaller measure as $A_2$, and repeat this process to create $A_i$. Note that the $A_i$s are measurable by induction.

Since at each step, $\prob(B_i) < \prob(A_i)$, $\prob(B_i), \prob(A_i \backslash B_i) > 0$. Thus, we construct a decreasing sequence $A_1 \supset A_2 \supset \dots$ such that at each step, $0 <\prob(A_i) \leq \frac{\prob(A)}{2^i}$. Eventually, for large enough $i$, we will get a $A_i$ with $\prob(A_i) < \epsilon$.

\subsubsection*{Part b}

Prove that if $0 < a < \prob(A)$, then there exists $B \subsetneq A$ with $\prob(B) = a$.


I will follow the hint used in the notes. Let $\epsilon_n \downarrow 0$ be arbitrary but fixed. Define inductively the numbers $x_n$ and sets $G_n, H_n \in \F$ as follows.

Let $H_0 = \emptyset$ and $H_n = \cup_{k<n} G_k$. What we're going to do is repeatedly take slices out of $A$. Denote these slices $G_i$. $H_n$ is the union of all the previous slices, and we keep taking slices $G_i$ to slowly increase $\prob(H_n)$ up to $a$, with $\epsilon_n$ controlling the rate of convergence. We construct these sets inductively. We start with the base case.

We first figure out what the 'maximum' possible slice is. Let $x_n = \sup\{\prob(G): G \subset A\backslash H_n, \prob(H_n \cup G) \leq a\}$. In other words, $x_n$ is limiting value of the possible $G$s at this $n$th step, such that $\prob(H_{n+1})$ will be $\leq a$.

This is where the non-atomic assumption comes in. $A \in \F$ by assumption, and $H_0 = \emptyset \in \F$ trivially. At the $n=1$ step, $A \backslash H_1 = A$ is measurable. By the assumption that $\prob$ is nonatomic and from Part a, there exists a $G \subset A$ such that $\prob(G) < a$. Thus this supremum is over a nonempty set, and $x_1$ is a real number.

We now determine the actual $G_1$ we take. Because $x_1$ is the supremum of possible $G_1$ subsets we can take, we can take a $G_1$ such that $\prob(G_1)$ is arbitrarily close to $x_1$. This is where the $\epsilon_1$ comes in; it controls how arbitrarily close to $x_1$ we want $\prob(G_1)$ to be.

Thus, let $G_1$ be a set $G_1 \subset A\backslash H_1 = A$ such that $\prob(H_1 \cup G_1) \leq a$ and $\prob(G_1) \geq (1-\epsilon_1)x_1$. As discussed above, this construction is well defined. We thus have a $G_1$ with the following properties:

\begin{itemize}
\item $\prob(H_1 \cup G_1) \leq a$
\item \textbf{SOMETHING ABOUT HOW THE G1 ARE GETTING CLOSER TO a}
\end{itemize}

\textbf{TODO}

inductively, $H_n \in \F$ for all $n \in \N$, implying $A\backslash H_n \in \F$ for all $n \in \N$. Now using the non-atomic assumption, there exists at $G \subset A\backslash H_n$ such that $\prob(G) < \prob(A\backslash H_n) = \prob(A) - \prob(H_n)$.

\end{document}