\documentclass{article}
\usepackage{amsmath}
\usepackage{amsfonts}
\usepackage{amssymb}

\newenvironment{proof}{\paragraph{Proof:}}{\hfill$\square$}
\newtheorem{theorem}{Theorem}
\newtheorem{lemma}[theorem]{Lemma}
\newtheorem{corollary}[theorem]{Corollary}

\newcommand{\A}{\mathcal{A}}
\newcommand{\F}{\mathcal{F}}
\newcommand{\prob}{\boldsymbol{P}}

\newcommand{\N}{\mathbb{N}}

\author{Arthur Chen}
\title{Chapter 1}
\date{\today}

\begin{document}
\maketitle

\section*{1.1 Probability spaces, measures, and $\sigma$-algebras}

\subsection*{Problem 1.1.4}

Let $(\Omega, \F, \prob)$ be a probability space and $A, B, A_i$ events in $\F$. Prove the following properties of every probability measure.

\subsubsection*{a}

Monotonicity: If $A \subset B$ then $\prob(A) \leq \prob(B)$.

If $A$ is not a proper subset of $B$, then $A=B$ and equality follows because $\prob$ is a function. Otherwise, $A$ partitions $B$ into $A$ and $B \backslash A$. Then by countable subadditivity,

\[
\prob(B) = \prob(A) + \prob(B \backslash A) \geq \prob(A)
\]

by the nonnegativity of measure.

\subsubsection*{b}

Sub-additivity:

\[
A \subset \cup_i A_i \longrightarrow \prob(A) \leq \sum_i \prob(A_i)
\]

Let $B_i = \cup_{k=1}^i A_k$, and $C_i = B_i - B_{i-1}$, letting $B_0 = \emptyset$. Then $\cup_i A_i = \cup_i C_i$, and $A \subset \cup_i C_i$. By monotonicity, $\prob(A) \leq \prob(\cup_i C_i)$. Since the $C_i$ are disjoint, $\prob(\cup_i C_i) = \sum_{i=1} \prob(C_i)$. Since $C_i = \cup_{j=1}^i A_j - \cup_{k=1}^{i-1}A_k$, $C_i \subset A_i$, so by monotonicity,

\[
\prob(A) \leq \sum_{i=1} \prob(C_i) \leq \sum_{i=1} \prob(A_i)
\]

\subsubsection*{c}

Continuity from Below:

$A_i \uparrow A$ implies $\prob(A_i) \uparrow \prob(A)$.

Note that $\prob(A_i)$ is weakly increasing.

Define $B_i = A_i - A_{i-1}$, and let $A_0 = \emptyset$. Then $A_i = \cup_{k=1}^i B_i$. Then by disjointedness $\prob(A_i) = \sum_{k=1}^i \prob(B_k)$ implies

\[
A = \cup_{i=1}^\infty = \cup_{i=1}^\infty \cup_{k=1}^i B_k = \cup_{i=1}^\infty
\]

and by countable additivity,

\[
\prob(A) = \sum_{k=1}^\infty \prob(B_k)
\]

Since the range of $\prob$ is $[0, 1]$, $|\prob(A)| < \infty$ and the series $\sum_{k=1}^\infty \prob(B_k)$ converges to $\prob(A)$. Since $A_i = \cup_{k=1}^i B_k$, $\prob(A_i)$ is the $i$th partial sum of the $\prob(B_k)$ series, which converges. Thus the $\prob(A_i)$ sequence converges to $\prob(A)$.

\subsubsection*{d}

Continuity from Above:

$A_i \downarrow A, \cap_i^\infty A_i = A$ implies $\prob(A_i) \downarrow A$.

\textbf{WORK IN PROGRESS}. I'm pretty sure I'm being sloppy when I switch from the unions/intersections in sets and the corresponding limits in probability. I need to fix that up.

We first begin with an obvious lemma.

\begin{lemma}
$\prob(B\backslash A) = \prob(B) - \prob(A)$
\begin{proof}
$A$ and $B\backslash A$ partition $B$.
\end{proof}
\end{lemma}

Let $B_i = A_1 \backslash A_i$. From the construction, we see that

\[
B_i \cup A \uparrow A_1
\]

Because $B_i$ and $A$ are disjoint, by continuity from below

\[
\prob(B_i) + \prob(A) \uparrow \prob(A_1)
\]

We now exploit this fact. Taking the union of all $B_i$'s to take advantage of continuity from below and using DeMorgan's laws,

\[
\cup_i^\infty B_i = \cup_i^\infty A_1 \backslash A_i = A_1 \backslash \cap_i^\infty A_i
\]

Taking probabilities on both sides and looking at the right side,

\[
\prob(A_1 \backslash \cup_i^\infty A_i) = \prob(A_1) - \lim_{n \rightarrow \infty} \prob(\cap_{n=1}^i A_i)
\]

Using the left side and continuity from below, we get

\[
\prob(A_1) - \prob(A) = \prob(A_1) - \lim_{n \rightarrow \infty} \prob(\cap_{n=1}^i A_i)
\]

which simplifies to the desired result. Downwards convergence then follows because $\prob$ being a nonnegative function and $\cap_{n=1}^i A_i$ being weakly decreasing imply that $\prob(\cap_{n=1}^i A_i)$ is weakly decreasing.

\subsection*{Problem 1.1.5}

Prove that a finitely additive non-negative set function $\mu$ on a measurable space $(\Omega, \F)$ with the 'continuity' property

\[
B_n \subset \F, B_n \downarrow \emptyset, \mu(B_n) < \infty \longrightarrow \mu(B_n) \rightarrow 0
\]

must be countably additive if $\mu(\Omega) < \infty$. Give an example that it is not necessarily so when $\mu(\Omega) = \infty$.

Let $A_n \in \F$ be a countable collection of disjoint sets. Since the $A_n$ are measurable, we can take $\mu(A_i)$.

\begin{lemma}
Let $A_n \in \F$ be a countable collection of disjoint measurable sets, and let $\mu$ be a non-negative finitely additive function. Then for all $n \in \N$,

\[
\mu(\cup_{i=1}^\infty A_i) = \sum_{i=1}^n \mu(A_i) + \mu(\cup_{i=n+1}^\infty A_i)
\]
\begin{proof}
The collection of sets $\{A_1, A_2 \dots A_n, \cup_{i=n+1}^\infty A_i\}$ is finite, and because sigma-algebras are closed under countable union, $\cup_{i=n+1}^\infty A_i \in \F$. The statement then follows because $\mu$ is finitely additive.
\end{proof}
\end{lemma}

\begin{theorem}
Let $M$ be the codomain of $\mu$ restricted on the collection $A = \{A_n\}$ - that is, $M$ is the set of values that $\mu$ takes when given $A_i$ as input. Then $M$ has a maximal value.
\begin{proof}
If $\mu(A_i)$ is nonzero only for a finite number of $A_i$'s, then the proof is trivial.

Suppose that $\mu(A_i) > 0$ for a countable number of $A_i$'s. Suppose that $\mu$ does not have a maximal value, and let $A_i$ be an arbitrary set such that $\mu(A_i) = \epsilon > 0$. Then there exists $A_j$ such that $\mu(A_i) < \mu(A_j)$, and $A_k$ such that $\mu(A_j) < \mu(A_k)$, and so forth. Thus by induction, there are a countable number of sets $A_j, A_k \dots$ such that $\mu(A_j) > \epsilon$. Denote this collection $B$.

$\mu(\Omega) < \infty$, so denote $\mu(\Omega) = M$. Since for all elements of $B$, $\mu(A_j) > \epsilon$, by the Archimedian property, we can take a finite number of the elements in $B$ such that the sum of their measures is greater than $M$.

\[
\mu(\cup_{i=1}^\infty A_i) = \sum_{i=1}^n \mu(A_i) + \mu(\cup_{i=n+1}^\infty A_i) > M
\]

by the above discussion and the non-negativity of $\mu$. However, $\cup_{i=1}^\infty A_i \subset \Omega$, so by the monotonicity of $\mu$, $\mu(\cup_{i=1}^\infty A_i) \leq M$. Thus by contradiction, $M$ has a largest element.
\end{proof}
\end{theorem}

TODO.

Relabel the $A_i$ so that they are ordered in decreasing measure. That is, $\mu(A_1) \geq \mu(A_2) \geq \dots$. We can do this by the above theorem because $M$ has a maximal element and $M$ is totally ordered.

(Formally, I might have to invoke Zorn's lemma here, but I am not strong enough with it to say.)

\end{document}